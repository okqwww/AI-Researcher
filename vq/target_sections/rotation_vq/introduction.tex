\section{Introduction}

Generative models have become fundamental in the realm of machine learning, with diverse applications in image generation, text synthesis, and more. Among these models, Vector Quantized Variational Autoencoders (VQ-VAEs) have attracted significant attention due to their capacity to effectively model complex data distributions using discrete latent variables. VQ-VAEs encode continuous data into finite representations, facilitating efficient storage and retrieval while preserving strong generative capabilities. The foundational work by van den Oord et al. \cite{vqvae} has established a robust framework for optimizing latent embeddings via a quantization process. This framework has been further enhanced by recent advances in categorical representation learning and variational inference techniques \cite{botta2020, gumbelsoftmax}.

Despite these advancements, current VQ-VAE implementations face considerable challenges, particularly regarding the gradient propagation through the non-differentiable quantization layer. The quantization process can impede the efficient flow of gradients required for effective learning, raising concerns such as codebook collapse, where many latent codes are underutilized. This underutilization compromises the model's ability to generate diverse samples, thereby limiting its practical effectiveness. Additionally, the complex interaction between quantization and gradient descent presents significant challenges that necessitate innovative solutions. This research endeavors to address these limitations by exploring optimal gradient transport and codebook management strategies within the VQ-VAE framework.

To tackle these challenges, we propose an Enhanced VQ-VAE architecture that incorporates a Rotation and Rescaling Transformation (RRT) to improve gradient transport across the quantization layers. The RRT is designed to preserve angular relationships within the latent space, thereby optimizing gradient backpropagation. Moreover, we introduce robust codebook management strategies aimed at enhancing code utilization throughout the training phase. Collectively, these components work synergistically to maintain effective gradient flow and support diverse latent representations, ultimately improving the generative performance of the model.

This work makes several significant contributions to the field:
\begin{itemize}
    \item We introduce a novel Enhanced VQ-VAE architecture that utilizes the Rotation and Rescaling Transformation to effectively address gradient flow issues.
    \item Our methodology for robust codebook management mitigates the risks of codebook collapse and underutilization, thus improving model performance.
    \item Empirical results demonstrate substantial enhancements in generative performance, evidenced by improved reliability and output quality through extensive experiments on standard datasets.
    \item We provide a comprehensive analysis of the interplay between quantization and gradient propagation, offering insights that pave the way for future research in generative modeling with discrete latent spaces.
\end{itemize}
